\documentclass{beamer}

\usepackage{amssymb}
\usepackage{amsfonts}
\usepackage{amsmath}
\usepackage{amsthm}
\usepackage{setspace}
\usepackage{longtable}
%already present \usepackage{enumerate}
\usepackage{mathtools}
\usepackage{color}
\usepackage{array}
\usepackage{calc} 
\usepackage{bm}
\usepackage{caption}
\usepackage{float}

\usetheme{Warsaw}
\usecolortheme{}
%rewrite default  color theme
\useinnertheme{rectangles}
%controls inner details change bullets to rectangles.
\useoutertheme{tree}
\useoutertheme{infolines} %gives numbering to slides
%controls outer detailes ex slide numbering title etc...
\usefonttheme{serif}
%rewrite font theme in warsaw
\setbeamertemplate{frame numbering}[fraction]
\setbeamercolor{background canvas}{bg=white}
\usepackage{multicol}

\title[subtitle]{Functions, Limits, Derivatives}
%[subtitle]change text on bottom bar
\subtitle{dfg}
%\subtitle{Subtitle Here}
\author[]{yyyyyyyyyy}
%in bottom nothing appear
\institute{yyyyy}
\date{vvv}

\setbeamercovered{transparent=7}

\begin{document}

\begin{frame}
\titlepage
\end{frame}

\begin{frame}[t]{Functions} \vspace{4pt} %spacing vertically
%by default text starts at middle if we want to change 
%it add [t] top
\begin{block}{Definition of a Function}
\vspace{0.5em}
A \textbf{function} $f$ is a rule that assigns to each element $x$ in a set $D$ exactly one element, called $f(x)$, in a set $E$.
\vspace{0.5em}
\end{block}
\end{frame}

\begin{frame}{sth}
sghshthrt \\ 
agsgsgg  \\ 
 \begin{exampleblock}{An example of typesetting tool}
        Example: MS Word, \LaTeX{}
    \end{exampleblock}
     \begin{alertblock}{Alert Message}
        This block presents alert message.
    \end{alertblock}
    
\end{frame}

\begin{frame}[t]{Fritle}
    \begin{enumerate}
        \item wefsg  uhguihg \\
        \item gsgs \\
    \end{enumerate}
\end{frame}

\end{document}